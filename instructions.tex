% Autogenerated translation of instructions.txt by Texpad
% To stop this file being overwritten during the typeset process, please move or remove this header

\documentclass{article}     \usepackage[utf8]{inputenc}     \begin{document}     1.	Rename the folder lastname1-lastname2-in.\\2.	Import the project in Eclipse (or another IDE, for example Netbeans).\\3.	Reference the library repast.\\4.	Other libraries such as colt.jar and plot.jar might be needed, so you can reference them as well. They can be downloaded on moodle (Additional JAR libraries for the programming exercises).\\5.	Link the javadoc for the repast library which can be found on moodle (Repast javadoc).\\6.	Write your code in RabbitsGrassSimulationAgent, RabbitsGrassSimulationModel and RabbitsGrassSimulationSpace (do not rename these files and do not put them into packages!).\\7.	Run the simulation by running MainRabbit.java.\\8.	Make sure that you fulfill the requirements for the solution given in the exercise description.\\9.	Write documentation using the latex template and place it into the doc folder. The pdf should be named lastname1-lastname2-in.pdf.\\10.	Create a runnable jar file and place it in the folder lastname1-lastname2-in.\\11.	Zip the folder lastname1-lastname2-in (without the libraries) and submit it on moodle.\\     \end{document}     